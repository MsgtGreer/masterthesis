%
% Electron-proton NC scattering
%
% Comment in the lines from \documentclass to \begin{document}
% and from \write18 to \end{document}
% to use the standalone package for testing the Feynman graphs
% You should also change the fmffile name to ep_cc-mp
% \documentclass{standalone}

% \usepackage{feynmp}
% \DeclareGraphicsRule{*}{mps}{*}{}

% \unitlength=1mm
% \begin{document}
\begin{fmffile}{ep_nc}
\begin{fmfgraph*}(50,50) \fmfpen{thin}
  \fmfleft{i1,i2} \fmfright{o1,o2}
  \fmfv{label=$X$,l.angle=10,l.dist=3*thick}{o1}
  \fmf{heavy,label=$p(P)$,l.side=left}{i1,v1}
  \fmf{plain}{v1,o1}
  \fmf{fermion,lab=$e(k)$,l.side=right}{i2,v2}
  \fmf{fermion,lab=$e^\prime(k^\prime)$}{v2,o2}
  \fmf{photon,lab=$\gamma,, Z(q)$,l.side=right}{v1,v2}
  \fmfv{decor.shape=circle,decor.filled=.5,decor.size=.20w}{v1}
  \fmfdot{v2}
  \fmffreeze
  \fmfi{plain}{vpath (__v1,__o1) shifted (thick*(0.7,2))}
  \fmfi{plain}{vpath (__v1,__o1) shifted (thick*(-1,-2))}
\end{fmfgraph*}
\end{fmffile}
% Include this command for Metapost
% \write18{mpost ep_nc-mp}
% \end{document}
