% !TEX root = mythesis.tex
% !TeX spellcheck = en_GB
%==============================================================================
\chapter{Introduction}
\label{sec:intro}
%==============================================================================
Modern, large scale physics experiments push the limits for achievable rates as well as precision and accuracy. One prominent example is the ongoing upgrade of the \ac{LHC} to achieve a ten times higher luminosity \cite{LHC}. To keep up with the update, the affiliated experiments at the \ac{CERN} are receiving an upgrade as well. As an example, \ac{ALICE} receives several upgrades. The inner tracking system and the muon forward tracker get replaced by new pixel sensor chips. For the \ac{TPC}, a new amplification and readout stage is required. Additionally, a new trigger system is set up. For details on the upgrade see \cite{alice}.

In order to produce reliable results with such experiments, the fundamentals of their behaviour need to be well understood and precisely quantized. Physicists all around the world conduct small-scale experiments with the hardware and collaborate in order to meet this goal. 
In the scope of these studies, specialized hardware is developed. One example of this hardware is the \ac{pAM}, originally developed at TU München \cite{bugl,roedel,rudolph}. It allows precision current measurements, in the picoampere regime, on high voltage lines. The \acp{pAM} are used for studies on micro-pattern gaseous detectors.
To keep up with the requirements of the large scale experiments, equipment as the \ac{pAM} need to be updated and improved regularly. 

In the scope of this thesis, a new revision of \acp{pAM} is developed. 
At first, in chapter \ref{sec:theory}, background informations necessary for this thesis are provided. In chapter \ref{sec:current_status}, an analysis of the existing design is conducted, revealing weak spots and possible tasks to be addressed. The next step is to find solutions, that could improve the design; which is addressed in chapter \ref{sec:problem_solving}. On the software side, a new firmware for the \ac{pAM} is developed, allowing a more flexible use of the devices. 
On the hardware side, the analysis mainly focusses on the \ac{ovp}. An improved \ac{ovp} is suggested and evaluated. These considerations lead to the conclusion that the problems could be overcome by a new input stage. Hence, a new front-end, based on a \ac{tia}, is developed and tested in chapter \ref{sec:newfrontend}. The main goal of the revision is to allow a higher readout rate and more precise measurements at the same time. 

A lot of different programs are part of this thesis. For the sake of simplicity, the full software is not printed here but is available on the AG Ketzer GitLab. The AltiumDesigner files for the different designs and first contributions to a documentation for the devices can be found there as well.
%%% Local Variables: 
%%% mode: latex
%%% TeX-master: "mythesis"
%%% End: 
