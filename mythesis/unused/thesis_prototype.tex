% !TEX root = mythesis.tex
% !TeX spellcheck = en_GB
%==============================================================================
\chapter{Prototype design}
\label{sec:prototype}
%==============================================================================
It comes in a 8 lead smd package, as depicted in figure \ref{fig:ADA4530}. The pin out is optimized for guarding the input pins. 
\begin{figure}
	\centering
	\includegraphics[width=.6\textwidth]{spacer.jpg}
	\caption{Pin configuration of the ADA5430\cite{ADA4530}}
	\label{fig:ADA4530}
\end{figure}

The pinout of the adc is depicted in figure \ref{fig:ltc2327:pinout}, the pins purpose can be found in figure \ref{fig:ltc2327:pinout}.
\begin{minipage}{\textwidth}
	\centering
	\begin{minipage}{0.29\textwidth}
		\includegraphics[width=\textwidth]{spacer.jpg}
		\captionof{figure}{Pin configuration of the LTC2327\cite{LTC2327}}
		\label{fig:ltc2327:pinout}
	\end{minipage}\begin{minipage}{0.7\textwidth}
		\begin{tabular}{lll}
			\hline
			PIN & name & purpose\\\hline
			1 	& V$_\text{DDLBYP}$ & Supply bypass pin \\
			2   & V$_\text{DD}$     & Power supply pin\\
			3,6,16 & GND & Ground\\
			4      & IN+ & Signal input\\
			5      & IN- & analog ground sense\\
			7      & REFBUF & buffer output\\
			8      & REFIN & external reference input\\
			9      & CNV & Trigger convert\\		
			10      & CHAIN & -\\
			11      & BUSY & -\\
			12      & RDL/SDI & -\\
			13      & SCK & Clock input\\
			14      & SDO & Data output\\
			15      & OV$_\text{DD}$ & Logick voltage level input\\\hline
		\end{tabular}
		\captionof{table}{Pin configuration of the LTC2327\cite{LTC2327}}
		\label{tab:ltc2327:pinout}
	\end{minipage}
\end{minipage}
The LTC2327 can be configured to operate with different input voltage ranges, depending on the voltage on the reference buffer pins according to table \ref{tab:ltc2327:input}
\begin{table}
	\centering
	\begin{tabular}{ccc}\hline
		\multicolumn{3}{c}{Normal operation}\\\hline
		REFIN & REFBUF & Input range \\
		\SI{2.048}{\volt} & \SI{4.096}{\volt} & \SI{\pm10.24}{\volt}\\
		&&\\
		\multicolumn{3}{c}{Override REFIN}\\\hline
		REFIN & REFBUF & Input range \\
		\SI{1.25}{\volt} & \SI{2.5}{\volt} & \SI{\pm6.25}{\volt}\\
		\SI{2.048}{\volt} & \SI{4.096}{\volt} & \SI{\pm10.24}{\volt}\\
		\SI{2.4}{\volt} & \SI{4.8}{\volt} & \SI{\pm12}{\volt}\\
		&&\\
		\multicolumn{3}{c}{Override REFBUF}\\\hline
		REFIN & REFBUF & Input range \\
		\SI{0}{\volt} & \SI{2.5}{\volt} & \SI{\pm6.25}{\volt}\\
		\SI{0}{\volt} & \SI{5}{\volt} & \SI{\pm12.5}{\volt}\\\hline
	\end{tabular}
	\caption{Differential input range of the LTC2327 depending on the reference pins}
	\label{tab:ltc2327:input}
\end{table}