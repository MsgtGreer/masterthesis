% !TEX root = mythesis.tex
% !TeX spellcheck = en_GB

%==============================================================================
\chapter{Conclusion and Outlook}
\label{sec:conclusion}
%==============================================================================+
The goal of this thesis was to further improve the \ac{pAM} developed at the TU München. Therefore, an analysis of the design was conducted, resulting in 4 main topics to handle:
\begin{enumerate}
	\item An upgrade of the firmware, to make it compatible with modern compilers.
	\item An improvement of the front-end, to eliminate the internal non-linearity.
	\item Setup of a reliable \ac{ovp}.
	\item Enabling a faster readout to achieve a better time resolution.
\end{enumerate}
In chapter \ref{sec:problem_solving}, the necessary changes to the firmware are introduced. The process was used to implement a bidirectional communication, allowing more flexible use of the device.
 
In the next step, the source of the observed non-linearity was investigated in chapter \ref{sec:nonlinearity}, leading to the need for a new \ac{ovp} circuit. An approach using diode clamping was suggested in chapter \ref{sec:solving:ovp}. Tests proved the approach to be suitable for implementation in the \ac{pAM}. Unfortunately, an easy insertion of the suggested \ac{ovp} was not possible without changes to the front-end part of the PCB.
As the layout of a new revision was necessary, the front-end concept was changed over completely. The shunt resistor based input stage was replaced by a \ac{tia} setup. To allow faster measurements, an \ac{adc} with a higher sampling rate was implemented. Layout and performance of the new front-end are discussed in chapter \ref{sec:newfrontend}. The new design achieved an estimated precision of \SI{20}{\femto\ampere} at a readout rate of \SI{7}{\hertz}. The bandwidth is limited to \SI{20}{\hertz}. The power consumption increased to a moderate \SI{7}{\milli\ampere}, with frequent spikes of up to \SI{50}{\milli\ampere}. \\

In the future, the readout rate could be further increased by reducing the number of values that are combined in one measurement or by moving the time-consuming calculations to the computer side. Further increase of the readout rate, would require a faster \ac{uC}.
If a higher bandwidth is desired, the front-end could be expanded, using a broadband \ac{tia} with lower amplification followed by a voltage amplification stage.
To accommodate for the increased power consumption, a new power supply should be set up, either utilizing an improvement of the photovoltaic powering developed by \cite{rudolph}, or based on batteries with a higher capacity. \\

For further investigation of the behaviour of the new design, a second iteration of the prototype was developed. The setup could not be finished, as the ADA4530 \ac{opamp} used in the new front-end, could not be delivered in time by the manufacturer. 
To ensure the reliability of the design, long term stability tests are necessary, as well as the study of possible temperature influences. A comparison with similar devices developed by \cite{zagreb} is supposed to be run as well. 
To get a good measurement of the achievable precision and accuracy, a new calibration reference is needed. The \ac{keithley}, that was used up to now, could not perform measurements with the necessary precision.


 