% !TEX root = mythesis.tex
% !TeX spellcheck = en_GB

%==============================================================================
\chapter{Development of a new front-end}
\label{sec:newfrontend}
%==============================================================================
Analysis of the design in chapter \ref{sec:problem_solving} led to the conclusion that a new revision is necessary. As modifications of the input stage were necessary, another option for a front-end setup should be evaluated here. A different type of measurement circuit should be capable to perform more precise measurements, allow faster readout and show a significantly lower temperature influence.
\section{Design of the Amplification Stage}
%The shut based input stage from the old design relies on a voltage drop 
Modern current meters no longer use shunt resistor based configurations as an input stage, but rely on a so-called \acl{tia} \cite{lowlvl}. Refer to figure \ref{fig:tia:sketch} for a sketch of the basic principle. It allows measuring without a burden voltage and a direct current to voltage conversion:
\begin{align}
\label{eq:tia}
U_\text{OUT}=-R_\text{F}I_\text{IN}
\end{align}
\begin{figure}
	\centering
	\includegraphics[width=.6\textwidth]{../../../figures/TIA/TIAsketch.pdf}
	\caption{Basic principle of a \ac{tia}.}
	\label{fig:tia:sketch}
\end{figure}
A \ac{tia} based front-end has some advantages over a common shunt resistor \cite{mortuza}. It allows precision measurements in the \ac{pA} range and is already widely used in the field \cite{zagreb}.
Making use of a \ac{tia} as the input stage will allow to build an \ac{ovp}, without having to drive diodes at reverse voltages, see section \ref{sec:tia:ovp}. Thus, reducing leakage currents and improving the accuracy.
\subsection*{Amplifier Selection}
When designing an amplification stage, the first step is to choose a suitable \ac{opamp}. For details on \acp{opamp}, see section \ref{sec:theory:amplifiers}.
Therefore an insight into the requirements is necessary.
\subsubsection*{Input Bias Current}
One of the critical parameters for a \ac{tia} front-end is the input bias current. The input bias current places an absolute lower limit on the achievable accuracy. The amplifier chosen should have an input current possible as low as possible. The input bias current should not exceed \SI{10}{\femto\amp} to allow accurate measurements of \SI{100}{\femto\ampere}.
\subsubsection*{Noise}
A total noise analysis of this stage can only be performed when the design is complete. The amplifier should have a voltage noise density in the regime of \SI[per-mode=symbol]{}{\nano\volt\per\sqhz}, in order to have a maximum of total noise of a few \SI{}{\micro\volt}. Current noise needs to be below \SI[per-mode=symbol]{1}{\femto\ampere\per\sqhz}, as it gets amplified along with the current that is to be measured.
%\subsubsection*{Slew rate}
%As one of the design goals was to allow a fast readout the slew rate should be high to meet this goal.
\subsection*{Offset Voltage}
The offset voltage of the amplifier needs to be significantly lower than the \ac{LSB} of the \ac{adc} chosen. A maximum of a few \SI{}{\micro\volt} with time and temperature is desired.
 
A comparison of different amplifiers suitable for this purpose is shown in table \ref{tab:opamps}.
\begin{table}
	\centering
	\begin{tabular}{lcccc}
		\hline
		\ac{opamp} & \multicolumn{2}{c}{Input bias current \SI[per-mode=symbol]{}{\per\femto\ampere}} & Offset voltage drift & Comment \\
				&	typical	 	& max											 & \SI[per-mode=symbol]{}{\per\micro\volt\per\degreeCelsius} & \\\hline
		ADA4530 \cite{ADA4530} & $<$\SI{1}{} 	& $\pm$\SI{20}{}								 & +0.13/-0.7 & offers guard buffer \\
		AD549L \cite{AD549} & 75 	& 100											 & 5 & offers offset voltage correction \\
 	 OPA128 \cite{OPA128} & 40 	& 75											 & 5 & offers offset voltage correction \\
		LMC6442 \cite{LMC6442}& 5 	& 4000											 & 0.4 & \\
		\hline
	\end{tabular}
\caption{Characteristics of different \acp{opamp}. The AD549L is used in the older \ac{pAM} designs; the OPA128 is its successor. The LMC6442 is used by \cite{zagreb}.}
\label{tab:opamps}
\end{table}
For the design of a prototype, the ADA4530 was chosen as it offers the best compromise of all the considered amplifiers. Its internal guard buffer allows the implementation of an appropriate guarding structure, see \ref{sec:theory:guarding}.
The ADA4530 is designed to be used as an electrometer amplifier. It offers a very low input offset drift over time and temperature as desired, see figure \ref{fig:ada4530:offsetdrift} for details. The datasheet provides a design guide for usage as a \ac{tia} to interface a photodiode, which can be used to deduce rules for the usage in the \ac{pAM}.

\begin{figure}
	\centering
	\begin{subfigure}{0.49\textwidth}
		\includegraphics[width=\textwidth,page=3]{../../../figures/ada4530/driftvstime.png}
		%\caption{}
		%\label{fig:pcb:backend}
	\end{subfigure}
	\begin{subfigure}{0.49\textwidth}
		\includegraphics[width=\textwidth,page=2]{../../../figures/ada4530/driftvstemperatur.png}
		%\caption{}
		%\label{fig:pcb:frontend}
	\end{subfigure}
	\caption{Drift of the ADA4530 offset-voltage drift vs time and temperature \cite{ADA4530}.}
	\label{fig:ada4530:offsetdrift}
\end{figure}
\subsection*{ADC Selection}
The next essential step is to choose a suitable \ac{adc}. To allow faster measurements as the revision 3 and 4 \ac{pAM}, a higher sampling rate is necessary, in the order of \SI{}{\kilo\hertz} is the goal. As bipolar measurements are desired, an \ac{adc} with a bipolar input range is necessary.
The LTC2327 was chosen here\footnote{Any information regarding performance and operation is taken from the datasheet \cite{LTC2327}.}, as it offers the best compromise of these parameters that could be found. It is a \SI{16}{\bit} \ac{SAR} \ac{adc}, see section \ref{sec:theory:adc}. The \ac{SAR} architecture was chosen as offers a good compromise of sampling speed and resolution. The LTC2327 has a true bipolar input, offers an adjustable voltage range, \SI{500}{\kilo\sps} maximum sampling frequency and non-linearity of less than one LSB. 

\section{Circuit Layout}
Combining the \ac{adc} and amplifier, the front-end can be set up. The first step is to choose suitable voltage levels for driving the components.
\subsection*{Supply Voltage Levels}
The supply voltage for the LTC2327 is fixed to be \SI{5}{\volt}. The ADA4530 can handle supply voltages ranging from \SI{\pm2.5}{\volt} to \SI{\pm8}{\volt}. For the design of a prototype a \SI{\pm5}{\volt} supply was chosen, as a \SI{5}{\volt} source was necessary for the \ac{adc}. With this supply voltage for the \ac{opamp}, the maximum output of the amplifier stage is close to \SI{5}{\volt}. The \ac{adc} range should accordingly be set to \SI{\pm6.25}{\volt} to ensure best overlap of signal and digitisation range. This input range digitised with 15-bits sets an \ac{LSB} voltage of \SI{190.73}{\micro\volt}. 
Knowing this, the next step is to set up the feedback loop for the amplifier stage considering noise, resolution and compensation.\\
\subsection*{Over-Voltage Protection}
\label{sec:tia:ovp}
The operational amplifier in this setup is sensible to over-voltages between the two inputs. In order to prevent this, \ac{ovp} diodes are introduced in this setup as displayed in figure \ref{fig:tia:ovp}. The resistors R are supposed to limit the current through the diodes in case of an over-voltage condition. A resistor value of \SI{100}{\kilo\ohm} is chosen here, limiting the current to \SI{10}{\milli\amp} in case of a \SI{1}{\kilo\volt} over-voltage.
In theory, the voltage between the amplifier inputs is zero, however, the diodes still should have as small as possible leakage currents because of the small offset voltages from the amplifier. In accordance with findings from chapter \ref{sec:solving:ovp}, the BC546B transistor is used here.
\begin{figure}
	\centering
	\includegraphics[width=0.6\textwidth]{../../../figures/TIA/TIAsketchwOVP.pdf}
	\caption{OVP diodes implemented in the \ac{tia} setup.}
	\label{fig:tia:ovp}
\end{figure}
\subsection*{Considerations on the Feedback Loop}
As introduced before, the output voltage is directly proportional to the feedback resistor and digitised in 15-bits. The resulting ranges and achievable resolutions for different feedback resistors are summarized in table \ref{tab:tia:RF}.
\begin{table}
	\centering
	\begin{tabular}{ccc}\hline
		R$_\text{F}$ & Input current range & LSB current \\\hline
		\SI{10}{\giga\ohm} & \SI{\pm500}{\pico\ampere} & \SI{19}{\femto\ampere}\\
		\SI{1}{\giga\ohm} & \SI{\pm5}{\nano\ampere} & \SI{190}{\femto\ampere}\\
		\SI{100}{\mega\ohm} & \SI{\pm50}{\nano\ampere} & \SI{1.9}{\pico\ampere}\\
		\SI{10}{\mega\ohm} & \SI{\pm500}{\nano\ampere} & \SI{19}{\pico\ampere}\\\hline
	\end{tabular}
\caption{Input current ranges and achievable resolution for different feedback resistors $R_\text{F}$, with a \SI{\pm5}{\volt} output range and an \ac{adc} input range of \SI{\pm6.25}{\volt} digitised in 15-bits.}
\label{tab:tia:RF}
\end{table}
These considerations, however, are only valid for a pure DC case, which has its limitations as noise generated by parts as resistors\footnote{For details of electronic noise refer to chapter \ref{sec:theory:noise}.} introduces AC components. Therefore, an insight into the frequency behaviour is necessary, requiring a model of the complete input stage. An approach to this is depicted in figure \ref{fig:tia:ltspicemodel}. It models capacitances between the two amplifier inputs combined into the shunt capacitance $C_\text{S}$, summarizing the parasitic capacitances from traces, \ac{ovp} diodes and the external components. The parasitic capacitance is relatively small. The contribution from external devices is hard to model, as it contains various factors. In \cite{zagreb} the shunt capacitance was estimated to be \SI{5}{\nano\farad}. For simulations, here an upper limit of \SI{10}{\nano\farad} is set.  The resistance between the inputs is modelled with $R_\text{S}$; the resistance between the two input paths through the PCB is the main contributor. It is estimated to be in the order of \SI{}{\tera\ohm} when the board is perfectly clean but gets significantly smaller when there are contaminations in between the traces. Any capacitance between input and output of the amplifier is modelled in $C_\text{F}$; it can also model an additional feedback capacitance introduced.
\begin{figure}
	\centering
	\includegraphics[width=0.8\textwidth]{../../../simulations/TIA/tia_completeNOISE/LTSPICEmodel.png}
	\caption{Model as implemented in LTspice for simulating the behaviour of the input stage. The Resistors R2 and R3 are the current limiters for the \ac{ovp} setup. Models for the BC546B are implemented in the transistors Q1 and Q2. C1 and R1 allow adjusting of shunt capacitance and resistance. C2 and R4 allow adjustment of the feedback impedance. For the ADA4530 a model is provided by Analog Devices.}
	\label{fig:tia:ltspicemodel}
\end{figure}
This model was used to run simulations on the feedback network.
\subsubsection*{Frequency Analysis}
In figure \ref{fig:tia:frequency:RF} different feedback resistors were simulated with both shunt and feedback capacitance set to \SI{0}{\farad}. The frequency response shows a peak at a frequency depending on the feedback resistance. This peak due to resonance of an RC-network formed by $R_\text{F}$ and the shunt capacitance $C_\text{S}$.
Figure \ref{fig:tia:frequency:CS} shows how the position of the resonance varies with $C_\text{S}$. These resonances are potentially dangerous and could destroy the amplifier or the \ac{adc}. The resonances can be compensated by adding a feedback capacitance. There are mathematical approaches to calculate $C_\text{F}$ depending on the \ac{opamp}, $R_\text{F}$ and $C_\text{S}$ \cite{tia_compensate}.
Figure \ref{fig:tia:frequency:CF} shows the combination of different feedback capacitances with different shunt capacitances. Leading to the conclusion, that for the choice of feedback capacitance both compensation and bandwidth need to be considered.
\begin{figure}
	\centering
	\begin{subfigure}{\textwidth}
		\centering
		\includegraphics[width=\textwidth,page=1]{../../../simulations/TIA/tia_completeAC/symAC_release.pdf}
		\caption{ Different values of R$_\text{F}$.}
		\label{fig:tia:frequency:RF}
	\end{subfigure}\hfill
	\begin{subfigure}{\textwidth}
		\centering
		\includegraphics[width=\textwidth,page=2]{../../../simulations/TIA/tia_completeAC/symAC_release.pdf}
		\caption{Different shunt capacitances.}
		\label{fig:tia:frequency:CS}
	\end{subfigure}
	\begin{subfigure}{\textwidth}
%	\begin{subfigure}{\textwidth}
		\centering
		\includegraphics[width=\textwidth,page=3]{../../../simulations/TIA/tia_completeAC/symAC_release.pdf}
		%\caption{The effect of different feedback capacities on the frequency behaviour of a transimpedance amplifier with a shunt capacitance of \SI{10}{\nano\farad}.}
%	\end{subfigure}\hfill
%	\begin{subfigure}{\textwidth}
%		\centering
%		\includegraphics[width=\textwidth,page=4]{../../../simulations/TIA/tia_completeAC/symAC_release.pdf}
%	\end{subfigure}
	\caption{Different feedback capacitances.}
	\label{fig:tia:frequency:CF}	
	\end{subfigure}
\caption{The effect of different parameters on the frequency behaviour of the \ac{tia} model. The y-axis shows the signal amplitude in \SI{}{\dB}. The current source is adjusted to achieve \SI{1}{\volt} DC output at $R_\text{F}=$\SI{5}{\giga\ohm}.}
\end{figure}
\subsubsection*{Noise Performance}
\label{sec:sim:nosie}
All these parameters influence the noise of the setup as well, see section \ref{sec:theory:noise}. The LTspice noise simulation determines the noise spectral density of a circuit. The above model was used to run noise simulations, again depending on the parameters introduced. One of the major contributors to the noise is the feedback resistance thermal noise. For a \SI{10}{\giga\ohm} resistor at \SI{25}{\degreeCelsius} it is around \SI[per-mode=symbol]{12.8}{\micro\volt\per\sqhz}, see also figure \ref{fig:tia:noise:RF}. The output signal (S) scales with $R_\text{F}$ (equation \ref{eq:tia}), the noise (N) of the resistor scales with $\sqrt{R}$ (equation \ref{eq:johnsonnoise}). Therefore the signal to noise ratio ($\nicefrac{S}{N}$) scales with $\sqrt{R}$, making a larger feedback resistance the preferred choice.\\
Another important factor in noise considerations is the feedback capacitance. As shown in figure \ref{fig:tia:noise:CF} a lower value of $C_\text{F}$ leads to lower overall noise. 
\begin{figure}
	\centering
	\begin{subfigure}{\textwidth}
		\includegraphics[width=\textwidth,page=1]{../../../simulations/TIA/tia_completeNOISE/NoiseAnalysis.pdf}
		\caption{Different feedback resistors.}
		\label{fig:tia:noise:RF}
	\end{subfigure}\hfill
	\begin{subfigure}{\textwidth}
		\includegraphics[width=\textwidth,page=2]{../../../simulations/TIA/tia_completeNOISE/NoiseAnalysis.pdf}
		\caption{Different shunt capacities for R$_\text{F}=$\SI{10}{\giga\ohm}.}
		\label{fig:tia:noise:CS}
	\end{subfigure}
	\begin{subfigure}{\textwidth}
	%\begin{subfigure}{0.49\textwidth}
		\centering
		\includegraphics[width=\textwidth,page=3]{../../../simulations/TIA/tia_completeNOISE/NoiseAnalysis.pdf}
		%\caption{Different feedback capacities for a shunt capacity of \SI{10}{\nano\farad}.}
	%\end{subfigure}\hfill
%	\begin{subfigure}{0.49\textwidth}
%		\centering
%		\includegraphics[width=\textwidth,page=5]{../../../simulations/TIA/tia_completeNOISE/symNOISE.pdf}
%		%\label{fig:tia:noise:CF2}
%	\end{subfigure}
	\caption{Different feedback capacities for a shunt capacity of \SI{10}{\pico\farad}.}
	\label{fig:tia:noise:CF}	
	\end{subfigure}
	\caption{The effect of different parameters on the spectral noise density. V$_\text{noise}$ denotes the integrated rms noise.}
\end{figure}
\\
In Figure \ref{fig:tia:noise:dep}, the influence of the shunt capacitance on the integrated rms noise is depicted. The \ac{adc} \ac{LSB} is only exceeded with C$_\text{F}\geq$\SI{2}{\nano\farad}.
For the prototype, the feedback capacitor was chosen to be \SI{2}{\pico\farad}, with a maximum feedback resistance of \SI{10}{\giga\ohm}. This setup should allow a bandwidth of $\approx$\SI{20}{\hertz}\footnote{The readout frequency achievable with this design is around \SI{7}{\hertz}, limited by the \ac{uC}, as discussed in section \ref{sec:results}. Hence a bandwidth \SI{20}{\hertz} is sufficient.} and an integrated rms noise of around \SI{60}{\micro\volt} for low values of $C_\text{F}$ and up to \SI{242}{\micro\farad} for higher values. 
This choice of feedback components offers a good compromise on stability, bandwidth and noise reduction.
\begin{figure}
	\centering
	\includegraphics[width=0.8\textwidth,page=1]{../../../simulations/TIA/tia_completeNOISE/NOISEdependencies.pdf}
	\caption{Noise scaling with the shunt capacitance. To allow comparison, the \ac{adc} LSB voltage is shown. It denotes the voltage corresponding to one \ac{adc} channel.}
	\label{fig:tia:noise:dep}
	%\caption{The effect of different parameters on the spectral noise density. V$_\text{noise}$ denotes the integrated rms noise}
\end{figure}
\section{Implementation of a Prototype}
\label{sec:proto:frontend}
The effects discussed above were used to design a prototype. The setup of the back-end circuitry was kept from older designs to ensure maximal compatibility with the available infrastructure. The firmware described in section \ref{sec:newfirmware} was modified to be compatible with the new components. The front-end was revised completely. The implementation of the prototype is discussed in the following paragraphs.
\subsection*{Supply Voltage Generation}
As stated previously, the LTC2327 requires a \SI{5}{\volt} supply, and the amplifier is planned to be driven from a \SI{\pm5}{\volt} supply. Additionally a \SI{1.25}{\volt} precision reference is necessary to set the \ac{adc} input range to \SI{6.25}{\volt}.
As the \ac{pAM} were so far driven by \SI{9}{\volt} block batteries, or a photovoltaic based supply delivering \SI{5.9\pm0.1}{\volt} \cite{rudolph}, the regulators generating the supply voltages should work with an input range of \SI{5.5} to \SI{10}{\volt}. The datasheet for the ADA4530 suggests using an ADP7118 LDO for the positive supply rail and an ADP7182 LDO for the negative supply rail as they provide a stable low noise voltage \cite{ADA4530}. The reference voltage to the \ac{adc} is generated with an LTC6655-1.25 as suggested by the datasheet.
\subsection*{Temperature Sensor}
As temperature was one of the weak spots in older designs, a temperature sensor was added to the design. For precision temperature measurements, an ADT7410 was chosen\footnote{Information on th ADT7410 are taken from the Datasheet \cite{ADT7410}.}. It offers a \SI{0.5}{\degreeCelsius} accuracy and low power consumption of \SI{7}{\micro\watt} in sleep mode. It has indicators for over- as well as under- temperatures (INT pin) and for critical temperature (CT pin). These threshold values can be programmed remotely. As the \acp{pAM} are not supposed to be exposed to extreme temperatures, these pins are not wired to the \ac{uC}. With the A0 and A1 pins, the address of the ADT can be set, as only one is used in the design, they are grounded. The setup was done as specified in the datasheet. For communication, the sensor uses the \ac{iic} protocol, see section \ref{sec:iic}. The ADT7410 can convert temperature either with 13 or 16-bits, the latter is chosen here. Measuring temperature is implemented in the mode signal blinking, in order to safe time, see the code snippet in listing \ref{lst:tempread}. 
To safe power and processing time, the signal blinking and temperature read is only run all 100 measurement cycles.
\begin{codecpp}[caption={The temperature reading and mode blinking loop\label{lst:tempread}.}]
if(CYCLE_COUNTER>=100){
	//blink \& read temperature every few seconds only, to safe power.
	temperature_sensor_write(0x03, 0xA0, 1);//start measurement,data ready in 240ms
	for (int i = 0; i < mode; i++) {
	\\blink to indicate the mode of operation
	WDTCTL = WDT_ARST_1000;
	debug_led_on();
	wait(300); 
	debug_led_off();
	wait(300);
	}
	//if reading not successful set temperature to 255
	if (!(temp = temperature_sensor_read(0x00, 2)))
		temperature=255;
	//convert \ac{adc} channels to degree celcius 
	if (temp > 0) {//positiv temperature
		temperature = temp / 128.; 
	} else {//negative temperature
		temperature = (temp - 65536.) / 128.;
	}
	CYCLE_COUNTER=0;
}
\end{codecpp}

\subsection*{Interfacing the \ac{adc}}
The pinout of the LTC2327 is summarized in table \ref{tab:ltc2327:pinout}. The table also shows the implemented connections. Over-driving the \pin{REFIN} pin with \SI{1.25}{\volt} sets the input range to be \SI{\pm6.25}{\volt}.
\begin{table}
	\centering
	\begin{tabular}[5pt]{llp{4cm}p{6.5cm}}
		\\\hline
		PIN & Name & Purpose & Implementation\\\hline
		1 	& \pin{V$_\text{DDLBYP}$} & Supply bypass pin & Bypassed with a \SI{2.2}{\micro\farad} ceramic capacitor \\
		2 & \pin{V$_\text{DD}$} & Power supply pin & \SI{5}{\volt} supply input\\
		3,6,16 & \pin{GND} & Ground &\\
		4 & \pin{IN+} & Signal input & Connected to \ac{opamp} output\\
		5 & \pin{IN-} & Analogue ground sense & Signal ground sensing. Connected to the groundplane.\\
		7 & \pin{REFBUF} & Internal reference buffer output & Bypassed with a \SI{47}{\micro\farad} ceramic capacitor\\
		8 & \pin{REFIN} & Internal Reference output / Input to reference buffer & Overdriven with a \SI{1.25}{\volt} reference. Bypassed with a \SI{100}{\nano\farad} ceramic capacitor\\
		9 & \pin{CNV} & Triggers convert & Connected to the \ac{uC}\\		
		10 & \pin{CHAIN} & Chain mode selector pin & Grounded as chain mode is not needed\\
		11 & \pin{BUSY} & Indicator for a running conversion & Connected to the \ac{uC} to serve as an interrupt signal\\
		12 & \pin{RDL/SDI} & Serial data input & Grounded, as it is not needed in normal operation\\
		13 & \pin{SCK} & Clock input & Connected to the \ac{uC} \\
		14 & \pin{SDO} & Serial data out & Connected to the \ac{uC}\\
		15 & \pin{OV$_\text{DD}$} & Logic voltage level input & Connected to the\SI{3.3}{\volt} regulator from the back-end\\\hline
	\end{tabular}
\captionof{table}{Pin configuration of the LTC2327 \cite{LTC2327}.}
\label{tab:ltc2327:pinout}
\end{table}
\begin{figure}
	\centering
	\includegraphics[width=\textwidth,page=5]{../../../figures/ltc2327/ltc2327_timing.png}
	\caption{Conversion timing characteristics of the LTC2327 \cite{LTC2327}. A conversion is started by pulling \pin{CNV} HIGH; it can be pulled LOW after $t_\text{CNVH}=$\SI{20}{\nano\second}. A conversion takes $t_\text{CONV}=$\SI{1.5}{\micro\second} and is indicated by \pin{BUSY} being HIGH. After a maximum of $t_\text{DSDOBUSYL}=$\SI{5}{\nano\second} data is valid for reading on \pin{SDO} and can be read out via the SPI interface.}
	\label{fig:ltc2327:timing}
\end{figure} 
An \ac{adc} conversion is started, when the \pin{CNV} pin is set high from the microcontroller, the LTC pulls the \pin{BUSY} pin high and converts data. On the falling edge of \pin{BUSY}, data is ready and can be read from \pin{SDO}. The timing diagram for this is depicted in figure \ref{fig:ltc2327:timing}. The original idea was to use the falling edge of \pin{BUSY} as an interrupt, as it was done in previous revisions. However, $t_\text{CONV}$ is only \SI{1.5}{\micro\second}, which is less than 3 clock ticks of the \ac{uC} and in almost all cases does not trigger an interrupt. To overcome this, a delay is implemented after starting a conversion, see listing \ref{lst:fimrwarenew:adc}.
\begin{codecpp}[caption={\ac{adc} conversion loop for taking 128 samples.\label{lst:fimrwarenew:adc}}]
for (pos = 0; pos < NUM ; /* pos++ in loop */) {
	WDTCTL = WDT_ARST_250;
	ADC_PORT_out set_high ADC_CNV;
	ADC_PORT_out set_low ADC_CNV;//trigger conversion
	__delay_cycles(10); //wait till conversion is ready
	mem[pos++] = clock_bits(0x0000,16)//reading 16 ADC bits 
	}
\end{codecpp} 
Here the internal function \code{\_delay\_cycles()} is used instead of \code{wait()}. The wait function uses a timer overflow interrupt to measure the time passed and does not work reliable with times less than \SI{1}{\milli\second}. \code{\_delay\_cycles()} has the disadvantage of higher power consumption but allows a much shorter time measurements. Still, the delay introduces an extra \SI{7}{\milli\second} runtime per full sample of 128 \ac{adc} readings. Lowering the cycle number below 10 sometimes results in problems with the \ac{adc} conversion. These timing issues could be overcome by choosing a faster clock speed on the \ac{uC} but would result in significantly higher power consumption. 
\subsection*{Amplifier \& Guard Structure}
For leakage reduction, the prototype relies on a guard ring. The guard structure surrounds both input traces as well as one side of the feedback resistors. Guarding structures do not offer high protection against leakage currents as ceramic stand-offs but are easier to implement and mechanically stable. Refer to section \ref{sec:theory:pecision} for details. The prototype is set up on a common \ac{FR-4} board; this can be changed for mass production. \\
The PCB layout is shown in figures \ref{fig:pcb:backend} and \ref{fig:pcb:frontend}; schematics can be found in appendix \ref{sec:app:proto1:schematics}.
%\begin{figure}
%	\centering
%	\includegraphics[width=.8\textwidth]{../../../figures/PCBprototype/PCBprototype.pdf}
%	\caption{Implementation of the input stage.}
%	\label{fig:pcb:inputstage}
%\end{figure}
The prototype was equipped with 3 slots for feedback resistors, selectable with relays. As finding suitable relays that are designed for low leakage currents proved to be an impossible task, they were placed after the feedback resistors in the voltage signal path. The remaining tracks form small antennas, but that is a disadvantage to be accepted.\\

The prototype was split into two boards to enable testing multiple front-ends. The back-end featuring the MSP430, the \acs{xbee} module and attached electronics was designed to allow maximum compatibility with possible front-end designs and therefore has most of the available \ac{uC} pins mapped to one boarder of the board. 
\begin{figure}
	\centering
	\begin{subfigure}{0.6\textwidth}
		\includegraphics[width=\textwidth]{../../../Figures/PCBprototype/BackendPCB.png}
		\caption{Back-end PCB designed to control different front-end concepts.}
		%\label{fig:pcb:backend}
	\end{subfigure}
	\begin{subfigure}{0.39\textwidth}
	 \begin{tabular}{lr}
	 	\hline
	 	Designator & Component \\\hline
	 	U104 & MSP430 \\
	 	U105 & XBee \\
	 	U106 & \SI{1.8}{\mega\hertz} crystal\\
	 	U108 & \SI{32}{\kilo\hertz} crystal\\
	 	CONN102 & JTag programmer interface\\
	 	CONN104 & Supply voltage input\\
	 	REL101 & Positive supply switch \\
	 	REL102 & negative supply switch\\
	 	S101 & Reed switch\\\hline 	
	\end{tabular}	%\label{fig:pcb:backend}
 	\caption{Components included in the back-end of the \ac{pAM} revision 5 prototype.}
	\end{subfigure}
	\caption{Back-end prototype. Schematics can be found in figure \ref{fig:schematics:backend}.}
	\label{fig:pcb:backend}
\end{figure}
\begin{figure}
	\begin{subfigure}{0.6\textwidth}
		\includegraphics[width=\textwidth]{../../../Figures/PCBprototype/FrontendPCB.png}
		\caption{Front-end prototype PCB, set up as described in section \ref{sec:proto:frontend}.}
		%\label{fig:pcb:frontend}
	\end{subfigure}
	\begin{subfigure}{0.39\textwidth}
	\begin{tabular}{lr}
		\hline
		Designator & Component \\\hline
		U101 & \SI{5}{\volt} regulator \\
		U102 & \SI{-5}{\volt} regulator \\
		U103 & \SI{1.25}{volt} reference\\
		U301 & ADA4530 \ac{opamp}\\
		U401 & LTC2327 \ac{adc}\\
		U501 & ADT7410 temperature sensor\\
		REL201 & \SI{10}{\giga\ohm} switch\\
		REL202 & \SI{1}{\giga\ohm} switch\\
		REL203 & \SI{100}{\mega\ohm} switch\\
		CONN201 & Input terminals\\\hline 	
		%\vfill
	\end{tabular}	%\label{fig:pcb:backend}
	\caption{Components included in the front-end of the \ac{pAM} revision 5 prototype.}
	\end{subfigure}
	\caption{Front-end prototype. Schematics can be found in figure \ref{fig:schematics:frontend}.}
	\label{fig:pcb:frontend}
\end{figure}
\subsubsection*{Shielding}
As explained in section \ref{sec:theory:pecision}, shielding is an important factor to reduce interference noise. For the construction of this prototype, no further shielding was implemented.
The device is operated in the calibration setup, which already features a Faraday box for shielding. The same goes for regular test setups, where the \ac{pAM} are operated.

For high-voltage supply of \acp{gem}, only the inner conductor is used. The outer conductor is connected to earth ground. Therefore, cables connecting the \ac{pAM} do not need extra shielding, as the outer conductor acts as a shield. 
\subsubsection*{Cleaning Procedure}
In section \ref{sec:theory:precision}, the importance of cleaning was discussed. Cleaning needs to be performed after the board is assembled. The relays slots must be left unequipped. The cleaning is performed according to the recommendations of the ADA4530 datasheet \cite{ADA4530}:
\begin{enumerate}
	\item Ultrasonic cleaning in isopropyl alcohol for \SI{30}{\minute}
	\item Flushing the board with isopropyl 
	\item Use a brush scrub the solder joints 
	\item Blow dry using compressed air
	\item Solder relays in place
	\item Repeat step 2, 3, and 4
	\item Bake the board at \SI{80}{\degreeCelsius} for \SI{3}{\hour}
\end{enumerate}
For further cleaning processes, the ultrasonic bath needs to be skipped. Baking should always be performed at the end of a cleaning process, as it reduces the moisture of board and components.
The relays need to be soldered in place after the ultrasonic bath, as they often get damaged in the process. An approach using sockets for the relays resulted in degraded precision.
\subsubsection*{Expected Temperature Behaviour}
Like all kinds of electronics, the above setup will experience a temperature influence. The expected influences of the components are shown in \ref{tab:tempdep}. Influences from the \ac{opamp} and the \ac{adc} are negligibly small. However, the influence of the feedback resistor over a full scale of \SI{\pm500}{\pico\ampere} results in an error of around \SI{3}{\pico\ampere}, over the full temperature range. The temperature of the device should, therefore, be observed while measuring. 
\begin{table}
	\centering
	\begin{tabular}{llcc}
	\hline				
	Component & Parameter & \multicolumn{2}{c}{Fluctuation} \\
	& & vs Temperature \SI[per-mode=symbol]{}{\per\degreeCelsius} & in the range of \SI{15} to \SI{30}{\degreeCelsius} \\\hline
	\multirow{2}{*}{ADA4530 \cite{ADA4530}}& Offset voltage & & \SI{\pm5}{\micro\volt} \\
										 & Input bias current & &\SI{\pm0.1}{\femto\ampere} \\
 \multirow{2}{*}{LTC2327 \cite{LTC2327}} & Non-Linearity & &\SI{\leq 1}{\ac{LSB}}\\
	 & Full-Scale Error& & \SI{\pm2}{\ac{LSB}} \\
	 & Offset Error && \SI{<<1}{}\ac{LSB}\\
	Feedback resistor (\SI{10}{\giga\ohm}) &Temperature Drift & \SI{\pm200}{ppm\per\degreeCelsius} & \SI{0.3}{\percent} \\
	\hline 	
	%\vfill
	\end{tabular}	%\label{fig:pcb:backend}
	\caption{Temperature dependencies of components in the front-end.}
	\label{tab:tempdep}
\end{table}
\section{Measurement Results}
\label{sec:results}
After construction of the prototype, as described above, first test measurements could be done. To allow measurements with the new front-end using the calibration station, a resistor had to be added in series between the voltage source and the \ac{pAM}. For the most sensitive mode, a \SI{2}{\giga\ohm} resistor was chosen. Combined with a \SI{\pm1}{\volt} voltage range, this yields a current range of \SI{50}{\nano\ampere}.
First calibration measurements produced the result displayed in figure \ref{fig:meas:1}. The resolution of \SI{0.67}{\pico\amp} is much higher, than the noise simulations hinted, see section \ref{sec:sim:nosie}. In figure \ref{fig:meas:1} on the right, the residuals plot shows large errorbars for the reference current measurement of the \ac{keithley}. The \ac{pAM} results, however, do not show significant errors.
\begin{figure}
	\centering
	\includegraphics[width=\textwidth,page=1]{../../../figures/pAMnewmeasurements/NewPloting/stat2/stat2_rmode2_2020-03-09_09-41.pdf}
	\caption{Calibration measurement done with the developed prototype. The obtained resolution is estimated to be \SI{0.67}{\pico\amp}.}
	\label{fig:meas:1}
\end{figure}
For both readings, the errors are the standard deviation of a sample of readings. To investigate the source of the large resolution determined in the calibration, the errors of \ac{keithley} and \ac{pAM} readings are plotted against the driving voltage, as shown in figure \ref{fig:meas:1:error}. The \ac{pAM} measurements have an average error of \SI{1.1\pm0.21}{\ac{adc} channels} \ac{adc} channels, which would correspond to a theoretical current error of \SI{20.1\pm4}{\femto\ampere}. In contrast to this, the current readings have an average measurement error of \SI{0.70\pm0.31}{\pico\ampere}. The \ac{keithley}, therefore, seems to be the limiting factor.
\begin{figure}
	\centering
	\begin{subfigure}[t]{\textwidth}
		\includegraphics[width=\textwidth,page=2]{../../../figures/pAMnewmeasurements/NewPloting/stat2/stat2_rmode2_2020-03-09_09-41.pdf}
	\end{subfigure}
	\begin{subfigure}[t]{\textwidth}
		\includegraphics[width=\textwidth,page=3]{../../../figures/pAMnewmeasurements/NewPloting/stat2/stat2_rmode2_2020-03-09_09-41.pdf}
	\end{subfigure}
	\caption{Comparison of the errors on the \ac{pAM} and the reference readings. Both are plotted against the voltage, that is used to generate the current to be measured.}
	\label{fig:meas:1:error}
\end{figure}

Better resolutions were achieved in calibrations of previous revisions, see figure \ref{fig:ovptest}. The difference between these measurements is (apart from the \ac{pAM} in use) the measurement time, that is available for the \ac{keithley}.
The changes to the CalibAna software, discussed in section \ref{sec:solving:calibana}, included a delay of 1 second in between conversions of the \ac{pAM}. With the older revisions, this delay was not necessary, as each conversion of a \ac{pAM} took eight seconds already.
A longer delay of eight seconds was therefore set in the software, to verify if the measuring time is a constraint here. The result of this measurement is shown in figure \ref{fig:meas:2}. The resolution improved in comparison with the previous measurement but is still not compatible with the \ac{pAM} performance.
\begin{figure} 
	\centering
	\includegraphics[width=\textwidth,page=1]{../../../figures/pAMnewmeasurements/NewPloting/stat2/stat2_rmode2_2020-03-10_14-28.pdf}
	\caption{Calibration measurement done with the developed prototype and an elongated measurement time for the \ac{keithley}.}
	\label{fig:meas:2}
\end{figure}
The average error on the \ac{pAM} reading is comparable with the previous measurement. The average current error improved to \SI{0.25\pm0.03}{\pico\ampere}. Further increasing the measurement time, only had a limited effect on the achievable resolution.

From the manual of the \acl{keithley} \cite{keithley}, it should be able to perform measurements with a precision of \SI{10}{\femto\ampere} in a range of \SI{\pm2}{\nano\ampere}. To verify, that this error is not caused by the \ac{pAM} inserted, a calibration measurement without \ac{pAM} was performed. The resulting current reading error is displayed in figure \ref{fig:meas:3:keyerr}. This measurement is in accordance with the previous results. Hinting, that using the \ac{keithley} to investigate on the characteristics is not sufficient for the new \ac{pAM} revision.
\begin{figure} 
	\centering
	\includegraphics[width=\textwidth,page=3]{../../../figures/pAMnewmeasurements/NewPloting/stat2/stat2_rmode2_2020-03-12_17-40.pdf}
	\caption{Calibration measurement done with the developed prototype and an elongated measurement time for the \ac{keithley}.}
	\label{fig:meas:3:keyerr}
\end{figure}
\subsection{Power Consumption}
The power consumption of the device plays an important role, as the devices need to operate on battery supply.
A measurement of the current draw led to the results shown in figure \ref{tab:frontend:new:power}.
\begin{table}[h!]
	\centering
	\begin{tabular}{lr}
		\hline
		 & Current \SI[per-mode=symbol]{}{\per\milli\amp}\\ \hline
		Without XBee & \SI{7\pm0.5}{} \\
		XBee sending & \SI{50\pm5}{} \\
		Total average & \SI{11\pm0.7}{}\\
		\hline
	\end{tabular}
	\captionof{table}{Current draw of the \ac{pAM} prototype. For the measurement, a \SI{10}{\ohm} resistor was inserted in the \SI{+9}{\volt} supply line, and the voltage drop was measured using an Oscilloscope.}
	\label{tab:frontend:new:power}
\end{table}
The power consumption increased significantly. The main contributor to the baseline power consumption is the LTC2327. According to the datasheet, it draws \SI{6}{\milli\ampere} in normal mode. The sleep mode of the \ac{adc} proved not to be suitable for the application, as wake up of the \ac{adc} takes \SI{200}{\milli\second}.
Another factor that contributes to an increased power consumption is the increased sampling rate, increasing the \acs{xbee} run time. 
The increased power demands, lower the run time with a \SI{9}{\volt} block battery to around 2 to 3 days.

Switching the power supply away from the \SI{9}{\volt} blocks could overcome the problem. A more durable supply could be build utilizing lithium-ion accumulators like the Samsung NR18650-25R. Two such accumulators in series can provide around \SI{7}{\volt}, which is sufficient to drive a \ac{pAM}. The large capacity allows a significantly larger runtime, as opposed to common \SI{9}{\volt} blocks. A comparison of both devices is shown in table \ref{tab:blockvslition}. Another option is, to update the photovoltaic based power supply, developed by \cite{rudolph}, to allow higher loads and an improved temperature performance.
\begin{table}
	\centering
	\begin{tabular}{lcc}
		\hline
		Parameter& \SI{9}{\volt} block \cite{blocks} & Samsung NR18650-25R \cite{samsung} \\ \hline
		Nominal Voltage & \SI{9}{\volt} & \SI{3.6}{\volt}\\
	 Capacity & \SI{950}{\milli\amp\hour} & \SI{2500}{\milli\amp\hour} \\
	 Benefit & Cheap and easily replaceable. & Rechargeable!\\
	 Downside & Has to be replaced regularly. & Requires protection circuit.\\		
		\hline
	\end{tabular}
	\captionof{table}{Comparison of a \SI{9}{volt}-block battery with a lithium-based accumulator.}
	\label{tab:blockvslition}
\end{table}

\subsection{Duty Cycle}
To determine the readout time of the \ac{pAM}, the \ac{adc} and \acs{xbee} operation was monitored, as shown in figure \ref{fig:meas:dutycycle}. On average between two transmission processes \SI{146.6\pm1.7}{\milli\second} pass, resulting in a readout frequency of $\approx$\SI{7}{\hertz}. From figure \ref{fig:meas:dutycycle}, the run time for different processes can be estimated. A conversion of 128 \ac{adc} values takes approximately \SI{36}{\milli\second}. The time between a transmission-block and a readout-block is to small too be resolved. The time between \ac{adc} conversion and conversion is around \SI{100}{\milli\second}, taking up most of the duty cycle. Between an \ac{adc} conversion and sending, the mean and standard deviation of the measured values is calculated, taking up most of the time. 
\begin{figure} 
	\centering
	\includegraphics[width=\textwidth,page=1]{../../../figures/PCandDC/symAC_release.pdf}
	\caption{Recording of the XBee \pin{TX} pin and the \ac{adc} \pin{SDO} pin. This measurement allows an estimation of the length of a duty cycle of the \ac{pAM}.}
	\label{fig:meas:dutycycle}
\end{figure}
Hence, the device readout frequency can be increased by a factor of 2 to 3, if the calculation of the mean and standard deviation is moved to the readout computer. This comes at the cost of more XBee run time, resulting in a higher power consumption.

Overall the prototype performed very well. Showing a high precision of \SI{20}{\femto\ampere} on its readings, no visible non-linear effects and a faster \SI{7}{\hertz} conversion with a reasonable power consumption. The limited precision of the reference did not allow further investigation on precision and accuracy.\\

For further evaluation of the concept, alike investigation of temperature influence, a more robust set of prototypes was designed and ordered. Changes to the design, PCB layout, and schematics can be found in the appendix \ref{app:secondprotoype}. However, due to shortages on the manufacturer side, the necessary \ac{opamp} is temporary not available. 